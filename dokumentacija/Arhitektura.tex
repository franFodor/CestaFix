\chapter{Arhitektura i dizajn sustava}

\textbf{\textit{dio 1. revizije}}\\

\textit{ Potrebno je opisati stil arhitekture te identificirati: podsustave, preslikavanje na radnu platformu, spremišta podataka, mrežne protokole, globalni upravljački tok i sklopovsko-programske zahtjeve. Po točkama razraditi i popratiti odgovarajućim skicama:}
\begin{itemize}
	\item 	\textit{izbor arhitekture temeljem principa oblikovanja pokazanih na predavanjima (objasniti zašto ste baš odabrali takvu arhitekturu)}
	\item 	\textit{organizaciju sustava s najviše razine apstrakcije (npr. klijent-poslužitelj, baza podataka, datotečni sustav, grafičko sučelje)}
	\item 	\textit{organizaciju aplikacije (npr. slojevi frontend i backend, MVC arhitektura) }
\end{itemize}

\noindent Arhitektura web aplikacije "CestaFix" može se podijeliti na tri podsustava:
\begin{itemize}
	\item 	\textit{Web poslužitelj}
	\item   \textit{Web aplikacija}
	\item   \textit{Baza podataka}
\end{itemize}

\noindent\textbf{Web poslužitelj} opis web poslužitelja
\newline
\textbf{Web preglednik} opis web preglednika
\newline
\textbf{Baza podataka} opis baze
\newline
\textbf{Front-end} opis fronte
\newline
\textbf{Back-end} opis backa



\section{Baza podataka}

\textbf{\textit{dio 1. revizije}}\\

\textit{Potrebno je opisati koju vrstu i implementaciju baze podataka ste odabrali, glavne komponente od kojih se sastoji i slično.\\} 
\noindent Sustav je temeljen na uporabi relacijske baze podataka implementirane u PostgreSQL-u gdje su entiteti modelirani kao tablice koje posjeduju svoje jedinstveno ime i skup atributa.
Odabir relacijske baze podataka proizlazi iz potrebe za lakšim ostvarenjem naših potreba za upravljanjem podacima pri prijavljivanju oštećenja i njihovoj sanaciji 
odnosno kako bismo što jednostavnije modelirali sustav prema stvarnom svijetu.
Ova baza podataka ključna je za sigurnost podataka i brz pristup, pohranu, umetanje, izmjenu te dohvat podataka koje sustav koristi za daljnju obradu.
Baza podataka ove aplikacija sadrži sljedeće entitete:
\begin{packed_item}
	\item \textit{Users}
	\item \textit{Reports}
	\item \textit{CityDep}
	\item \textit{Category}
	\item \textit{Problems}
	\item \textit{CitydepCategory}
\end{packed_item}



\subsection{Opis tablica}


\textit{Svaku tablicu je potrebno opisati po zadanom predlošku. Lijevo se nalazi točno ime varijable u bazi podataka, u sredini se nalazi tip podataka, a desno se nalazi opis varijable. Svjetlozelenom bojom označite primarni ključ. Svjetlo plavom označite strani ključ}

\noindent\textbf{Users} je entitet...
\begin{longtblr}[
	label=none,
	entry=none
	]{
	width = \textwidth,
	colspec={|X[6,l]|X[6, l]|X[20, l]|},
	rowhead = 1,
	} %definicija širine tablice, širine stupaca, poravnanje i broja redaka naslova tablice
	\hline \SetCell[c=3]{c}{\textbf{Users}}                                           \\ \hline[3pt]
	\SetCell{LightGreen}user\_id & INT     & Jedinstveni identifikator korisnika      \\ \hline
	name                         & VARCHAR & ime korisnika                            \\ \hline
	email                        & VARCHAR & e-mail adresa korisnika                  \\ \hline
	password                     & VARCHAR & hash lozinke korisnika                   \\ \hline
	role                         & VARCHAR & uloga korisnika                          \\ \hline
	citydep\_id                  & INT     & jedinstveni identifikator gradskog ureda \\ \hline
\end{longtblr}


\noindent\textbf{Reports} je entitet...
\begin{longtblr}[
	label=none,
	entry=none
	]{
	width = \textwidth,
	colspec={|X[13,l]|X[6, l]|X[20, l]|},
	rowhead = 1,
	} %definicija širine tablice, širine stupaca, poravnanje i broja redaka naslova tablice
	\hline \SetCell[c=3]{c}{\textbf{Reports}}                                                                \\ \hline[3pt]
	\SetCell{LightGreen}report\_id  & INT       & jedinstveni identifikator prijave                          \\ \hline
	\SetCell{LightBlue} user\_id    & INT       & jedinstveni identifikator korisnika (users.user\_id)       \\ \hline
	title                           & VARCHAR   & naziv prijave/oštećenja                                    \\ \hline
	description                     & TEXT      & opis oštećenja                                             \\ \hline
	address                         & VARCHAR   & adresa oštećenja                                           \\ \hline
	photo                           & BYTEA     & slika oštećenja                                            \\ \hline
	report\_time                    & TIMESTAMP & vrijeme podnošenja prijave                                 \\ \hline
	status                          & VARCHAR   & status prijave                                             \\ \hline
	\SetCell{LightBlue} problem\_id & INT       & jedinstveni identifikator oštećenja (problems.problem\_id) \\ \hline
\end{longtblr}

\noindent\textbf{Citydep} je entitet...
\begin{longtblr}[
	label=none,
	entry=none
	]{
	width = \textwidth,
	colspec={|X[6,l]|X[6, l]|X[20, l]|},
	rowhead = 1,
	} %definicija širine tablice, širine stupaca, poravnanje i broja redaka naslova tablice
	\hline \SetCell[c=3]{c}{\textbf{Citydep}}                                            \\ \hline[3pt]
	\SetCell{LightGreen}citydep\_id & INT     & jedinstveni identifikator gradskog ureda \\ \hline
	citydep\_name                   & VARCHAR & naziv gradskog ureda                     \\ \hline
\end{longtblr}

\noindent\textbf{Category} je entitet...
\begin{longtblr}[
	label=none,
	entry=none
	]{
	width = \textwidth,
	colspec={|X[6,l]|X[6, l]|X[20, l]|},
	rowhead = 1,
	} %definicija širine tablice, širine stupaca, poravnanje i broja redaka naslova tablice
	\hline \SetCell[c=3]{c}{\textbf{Category}}                                                  \\ \hline[3pt]
	\SetCell{LightGreen}category\_id & INT     & jedinstveni identifikator kategorije oštećenja \\ \hline
	category\_name                   & VARCHAR & naziv kategorije oštećenja                     \\ \hline
\end{longtblr}

\noindent\textbf{Problems} je entitet...
\begin{longtblr}[
	label=none,
	entry=none
	]{
	width = \textwidth,
	colspec={|X[6,l]|X[6, l]|X[20, l]|},
	rowhead = 1,
	} %definicija širine tablice, širine stupaca, poravnanje i broja redaka naslova tablice
	\hline \SetCell[c=3]{c}{\textbf{Problems}}                                                                           \\ \hline[3pt]
	\SetCell{LightGreen}problem\_id  & INT     & jedinstveni identifikator oštećenja                                     \\ \hline
	longitude                        & DOUBLE  & geografska dužina lokacije oštećenja                                    \\ \hline
	latitude                         & DOUBLE  & geografska širina lokacije oštećenja                                    \\ \hline
	status                           & VARCHAR & status oštećenja                                                        \\ \hline
	\SetCell{LightBlue} category\_id & INT     & jedinstveni identifikator kategorije oštećenja  (category.category\_id) \\ \hline
\end{longtblr}


\noindent\textbf{CitydepCategory} je entitet...
\begin{longtblr}[
	label=none,
	entry=none
	]{
	width = \textwidth,
	colspec={|X[6,l]|X[6, l]|X[20, l]|},
	rowhead = 1,
	} %definicija širine tablice, širine stupaca, poravnanje i broja redaka naslova tablice
	\hline \SetCell[c=3]{c}{\textbf{CitydepCategory}}                                                               \\ \hline[3pt]
	\SetCell{LightGreen}citydep\_id  & INT & jedinstveni identifikator gradskog ureda (citydep.citydep\_id)         \\ \hline
	\SetCell{LightGreen}category\_id & INT & jedinstveni identifikator kategorije oštećenja (category.category\_id) \\ \hline
\end{longtblr}



\subsection{Dijagram baze podataka}
\textit{ U ovom potpoglavlju potrebno je umetnuti dijagram baze podataka. Primarni i strani ključevi moraju biti označeni, a tablice povezane. Bazu podataka je potrebno normalizirati. Podsjetite se kolegija "Baze podataka".}

\eject


\section{Dijagram razreda}

\textit{Potrebno je priložiti dijagram razreda s pripadajućim opisom. Zbog preglednosti je moguće dijagram razlomiti na više njih, ali moraju biti grupirani prema sličnim razinama apstrakcije i srodnim funkcionalnostima.}\\

\textbf{\textit{dio 1. revizije}}\\

\textit{Prilikom prve predaje projekta, potrebno je priložiti potpuno razrađen dijagram razreda vezan uz \textbf{generičku funkcionalnost} sustava. Ostale funkcionalnosti trebaju biti idejno razrađene u dijagramu sa sljedećim komponentama: nazivi razreda, nazivi metoda i vrste pristupa metodama (npr. javni, zaštićeni), nazivi atributa razreda, veze i odnosi između razreda.}\\

\textbf{\textit{dio 2. revizije}}\\

\textit{Prilikom druge predaje projekta dijagram razreda i opisi moraju odgovarati stvarnom stanju implementacije}



\eject

\section{Dijagram stanja}


\textbf{\textit{dio 2. revizije}}\\

\textit{Potrebno je priložiti dijagram stanja i opisati ga. Dovoljan je jedan dijagram stanja koji prikazuje \textbf{značajan dio funkcionalnosti} sustava. Na primjer, stanja korisničkog sučelja i tijek korištenja neke ključne funkcionalnosti jesu značajan dio sustava, a registracija i prijava nisu. }


\eject

\section{Dijagram aktivnosti}

\textbf{\textit{dio 2. revizije}}\\

\textit{Potrebno je priložiti dijagram aktivnosti s pripadajućim opisom. Dijagram aktivnosti treba prikazivati značajan dio sustava.}

\eject
\section{Dijagram komponenti}

\textbf{\textit{dio 2. revizije}}\\

\textit{Potrebno je priložiti dijagram komponenti s pripadajućim opisom. Dijagram komponenti treba prikazivati strukturu cijele aplikacije.}