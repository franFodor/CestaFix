\chapter{Implementacija i korisničko sučelje}
		
		
		\section{Korištene tehnologije i alati}
		
			\textbf{\textit{dio 2. revizije}}
			
			 \textit{Detaljno navesti sve tehnologije i alate koji su primijenjeni pri izradi dokumentacije i aplikacije. Ukratko ih opisati, te navesti njihovo značenje i mjesto primjene. Za svaki navedeni alat i tehnologiju je potrebno \textbf{navesti internet poveznicu} gdje se mogu preuzeti ili više saznati o njima}.
			 
			 Komunikacija u timu realizirana je korištenjem aplikacije \textbf{\href{https://discord.com/}{Discord}}. 
			 Discord je društvena platforma na kojoj korisnici imaju mogućnost komuniciranja tekstualnim porukama, glasovnim
			 pozivima, videopozivima, medijima i datotekama u privatnim porukama ili kao dio zajednice koju nazivaju server. 
			 Za izradu UML dijagrama korišten je alat \textbf{\href{https://www.visual-paradigm.com/}{Visual Paradigm}}. Visual 
			 Paradigm je grafički alat koji omogućava jednostavno modeliranje mnogo različitih tipova UML dijagrama, poput 
			 dijagrama obrazaca uporabe, sekvencijskih dijagrama, dijagrama razreda, dijagrama stanja, dijagrama aktivnosti, 
			 dijagrama komponenata, ERD modela baze podataka i mnogih drugih. Za upravljanje izvornim kodom korišten je 
			 \textbf{\href{https://git-scm.com/}{Git}}. Git je open-source distribuirani sustav za upravljanje različitim 
			 verzijama datoteka. Udaljeni repozitorij projekta je dostupan na web platformi \textbf{\href{https://github.com/}{GitHub}}.
			 GitHub pruža usluge spremanja i upravljanja kodom. Koristi se Git-om kako bi omogućio upravljanje različitim 
			 verzijama datoteka. Također, GitHub omogućava dokumentiranje programske podrške pomoću wiki-ja.

			 Kao razvojno okruženje korišteni su \textbf{\href{https://code.visualstudio.com/}{Visual Studio Code}} 
			 i \textbf{\href{https://www.jetbrains.com/idea//}{Intellij IDEA}}. Visual Studio Code je uređivač teksta
			 razvijen u tvrtki Microsoft. Prvenstveno se koristi za razvoj računalnih sustava na operacijskom sustavu Windows. 
			 Korišten je za razvoj programske podrške na frontendu i razvoj dokumentacije. Intellij IDEA je integrirano razvojno 
			 okruženje (IDE) razvijeno u tvrtki JetBrains. Usmjereno je na razvoj Java aplikacija, no podržava niz drugih jezika i 
			 tehnologija. Korišten je za razvoj programske podrške na backendu.

			 Aplikacija je napisana koristeći radni okvir \textbf{\href{https://spring.io/projects/spring-boot}{Spring Boot}} i jezik 
			 \textbf{\href{https://www.java.com/en/}{Java}} za izradu \textit{backenda} te jezik \textbf{\href{https://www.javascript.com/}{JavaScript}} 
			 i njegovu biblioteku \textbf{\href{https://react.dev/}{React}} za izradu \textit{frontenda}. React je biblioteka 
			 u JavaScriptu za izgradnju korisničkih sučelja. Nastala je od strane Facebooka. Glavna karakteristika Reacta je komponentna 
			 arhitektura, što znači da se korisničko sučelje sastoji od više manjih, ponovno uporabljivih komponenata. Izrada 
			 složenijih aplikacija u Reactu obično zahtjeva korištenje dodatnih biblioteka za interakciju s API-jem. Radni okvir 
			 Spring Boot nudi gotova rješenja i funkcionalnosti koje ubrzavaju razvoj aplikacija. Ima automatsko upravljanje 
			 konfiguracijom i zavisnostima što olakšava i ubrzava posao programerima. Spring Boot pruža podršku za implementaciju 
			 sigurnosti u aplikacijama pomoću Spring Security modula. 

			 Baza podataka izvedena je u \textbf{\href{https://www.postgresql.org/}{PostgreSQL}}-u. PostgreSQL je open-source sustav za upravljanje relacijskim bazama
			 podataka kojim se proširuje funkcionalnost SQL-a. Dizajniran je da izdrži različita radna opterećenja, od 
			 pojedinačnih računala, pa sve do skladišta podataka ili web usluga s mnogo istodobnih korisnika. Baza podataka
			 se na poslužitelju u oblaku \textbf{\href{https://render.com/}{Render}}. Kao okruženje za upravljanje bazom
			 podataka korišten je open-source grafički alat \textbf{\href{https://www.pgadmin.org/}{pgAdmin}}.

		     Dokumentacija je pisana u jeziku \textbf{\href{https://www.latex-project.org/}{LaTeX}}. LaTeX je jezik za pisanje
			 strukturiranih tekstova profesionalne kvalitete. Za razliku od nekih programskih jezika za obradu teksta s grafičkim
			 sučeljem poput Microsoft Worda, dokumenti u LaTeX-u pisani su kao obični tekst s dodanom semantičkom strukturom. Time 
			 postiže usredotočenost na sadržaj, ujednačenost izgleda te brži i stabilniji rad.

			\eject 
		
	
		\section{Ispitivanje programskog rješenja}
			
			\textbf{\textit{dio 2. revizije}}\\
			
			 \textit{U ovom poglavlju je potrebno opisati provedbu ispitivanja implementiranih funkcionalnosti na razini komponenti i na razini cijelog sustava s prikazom odabranih ispitnih slučajeva. Studenti trebaju ispitati temeljnu funkcionalnost i rubne uvjete.}
	
			
			\subsection{Ispitivanje komponenti}
			\textit{Potrebno je provesti ispitivanje jedinica (engl. unit testing) nad razredima koji implementiraju temeljne funkcionalnosti. Razraditi \textbf{minimalno 6 ispitnih slučajeva} u kojima će se ispitati redovni slučajevi, rubni uvjeti te izazivanje pogreške (engl. exception throwing). Poželjno je stvoriti i ispitni slučaj koji koristi funkcionalnosti koje nisu implementirane. Potrebno je priložiti izvorni kôd svih ispitnih slučajeva te prikaz rezultata izvođenja ispita u razvojnom okruženju (prolaz/pad ispita). }
			
			
			
			\subsection{Ispitivanje sustava}

			Ispitiavanje funkcionalnosti sustava je provedeno pomoću Selenium tekstova
			pisanih i jeziku JavaScript uz preglednik Google Chrome. Ukupno je napisano
			8 testova od kojih 3 provjerava kako se sustav nosi sa pogrešnim unosima.
			 \\
			\textbf{1. Ispitni slučaj: Prijava korisnika}
			 \begin{itemize}
				\item \textbf{Ulaz:} Email adresa i lozinka korisnika 
				\item \textbf{Očekivani izlaz:} Nakon prijave nalazit ćemo se na početnoj stranici sa korisničkim imenom prikazanim u gornjem desnom kutu.
				\item \textbf{Koraci:} 
				\\ 1. Klik na gumb Login/Register
				\\ 2. Unos korisničnih podataka, točnije emaila i lozinke
				\\ 3. Klik na gumb Prijava
			\end{itemize}

			\begin{verbatim}
				const { By, Key, Builder, until } = require("selenium-webdriver");

				async function prijava() {
				let driver = await new Builder().forBrowser("chrome").build();
				try {
					await driver.get("https://cestafix-fe.onrender.com/");
					const buttonElement = await driver.findElement(By.css('#login'));
					await buttonElement.click();
					const usernameField = await driver.findElement(By.css('#username'));
					await usernameField.sendKeys('filip.simunovic@gmail.com');
					const passwordField = await driver.findElement(By.css('#password'));
					await passwordField.sendKeys('Simiklimi.5');
					const submitButton = await driver.findElement(By.css('#submit'));
					await submitButton.click();
					await driver.sleep(10000);
				} catch (error) {
					console.error("Test failed:", error);
				} finally {
					// Quit the driver
					console.log("Test je prosao uspjesno.");
					await driver.quit();
				}
				}

				prijava();
			\end{verbatim}

			\textbf{2. Ispitni slučaj: Registracija korisnika}
			 \begin{itemize}
				\item \textbf{Ulaz:} Ime, prezime, email i lozinka korisnika
				\item \textbf{Očekivani izlaz:} Nakon uspješne registracije nalazit ćemo se na početnoj stranici sa korisničkim imenom prikazanim u gornjem desnom kutu.
				\item \textbf{Koraci:} 
				\\ 1. Klik na gumb Login/Register
				\\ 2. Klik na link "Nemaš račun? Registriraj se!"
				\\ 3. Unos korisničnih podataka
				\\ 4. Klik na gumb Registriraj se
			\end{itemize}

			\begin{verbatim}
				const { By, Key, Builder, until } = require("selenium-webdriver");


				async function register() {
					let driver = await new Builder().forBrowser("chrome").build();
				  try {
					await driver.get("https://cestafix-fe.onrender.com/");
					const buttonElement = await driver.findElement(By.css('#login'));
					await buttonElement.click();
					const registerElement = await driver.findElement(By.css('#nemaracun'));
					await registerElement.click();
					const imeField = await driver.findElement(By.css('#imeid'));
					await imeField.sendKeys('Da');
					const prezimeField = await driver.findElement(By.css('#prezimeid'));
					await prezimeField.sendKeys('Vinki');
					const emailField = await driver.findElement(By.css('#emailid'));
					await emailField.sendKeys('da.vinki@gmail.com');
					const passwordField = await driver.findElement(By.css('#passwordid'));
					await passwordField.sendKeys('Vinki.66');
					const passwordrepeatField = await driver.findElement(By.css('#repatpasswordid'));
					await passwordrepeatField.sendKeys('Vinki.66');
					const submitField = await driver.findElement(By.css('#signupid'));
					await submitField.click();
					await driver.sleep(10000);
				  } catch (error) {
					console.error("Test failed:", error);
				  } finally {
					// Quit the driver
					console.log("Test je prosao uspjesno.");
					await driver.quit();
				  }
				
				}
				
				register();				
			\end{verbatim}

			\textbf{3. Ispitni slučaj: Neuspjela prijava korisnika}
			 \begin{itemize}
				\item \textbf{Ulaz:} Email adresa i \textit{netočna} lozinka korisnika
				\item \textbf{Očekivani izlaz:} Nakon unosa pogrešne lozinke korisniku se trebaju ispisati da su uneseni krivi podatci u istom prozoru.
				\item \textbf{Koraci:} 
				\\ 1. Klik na gumb Login/Register
				\\ 2. Unos korisničnih podataka, točnije emaila i lozinke
				\\ 3. Klik na gumb Prijava
			\end{itemize}

			\begin{verbatim}
				const { By, Key, Builder, until } = require("selenium-webdriver");

				async function prijava() {
				let driver = await new Builder().forBrowser("chrome").build();
				try {
					await driver.get("https://cestafix-fe.onrender.com/");
					const buttonElement = await driver.findElement(By.css('#login'));
					await buttonElement.click();
					const usernameField = await driver.findElement(By.css('#username'));
					await usernameField.sendKeys('filip.simunovic@gmail.com');
					const passwordField = await driver.findElement(By.css('#password'));
					await passwordField.sendKeys('Simiklimiii.5');
					const submitButton = await driver.findElement(By.css('#submit'));
					await submitButton.click();
					await driver.sleep(10000);
				} catch (error) {
					console.error("Test failed:", error);
				} finally {
					// Quit the driver
					console.log("Test je prosao uspjesno.");
					await driver.quit();
				}
				}

			prijava();
			\end{verbatim}

			\textbf{4. Ispitni slučaj: Registracija korisnika i brisanje računa}
			 \begin{itemize}
				\item \textbf{Ulaz:} Ime, prezime, email i lozinka korisnika
				\item \textbf{Očekivani izlaz:} Nakon registriranja i brisanja računa, ako se pokušamo logirati sa istim podatcima dobit ćemo neuspjelu prijavu.
				\item \textbf{Koraci:} 
				\\ 1. Klik na gumb Login/Register
				\\ 2. Klik na link "Nemaš račun? Registriraj se!"
				\\ 3. Unos korisničnih podataka
				\\ 4. Klik na gumb Registriraj se
				\\ 5. Klik na gumb koji prikazuje ime korisnika
				\\ 6. Klik na gumb Pobriši račun!!!
				\\ 7. Klik na gumb POTVRDI
				\\ 8. Klik na gumb Login/Register
				\\ 9. Unos korisničnih podataka
				\\ 10. Klik na gumb Prijava
			\end{itemize}

			\begin{verbatim}
				const { By, Key, Builder, until } = require("selenium-webdriver");

				async function register() {
					let driver = await new Builder().forBrowser("chrome").build();
				try {
					await driver.get("https://cestafix-fe.onrender.com/");
					const buttonElement = await driver.findElement(By.css('#login'));
					await buttonElement.click();
					const registerElement = await driver.findElement(By.css('#nemaracun'));
					await registerElement.click();
					const imeField = await driver.findElement(By.css('#imeid'));
					await imeField.sendKeys('Da');
					const prezimeField = await driver.findElement(By.css('#prezimeid'));
					await prezimeField.sendKeys('Vinki');
					const emailField = await driver.findElement(By.css('#emailid'));
					await emailField.sendKeys('da.vinkiiiiiiiii@gmail.com');
					const passwordField = await driver.findElement(By.css('#passwordid'));
					await passwordField.sendKeys('Vinkii.66');
					const passwordrepeatField = await driver.findElement
					(By.css('#repatpasswordid'));
					await passwordrepeatField.sendKeys('Vinkii.66');
					const submitField = await driver.findElement(By.css('#signupid'));
					await submitField.click();
					await driver.sleep(5000);
					const accountField = await driver.findElement(By.css("#account"));
					await accountField.click();
					await driver.sleep(2000);
					const deleteField = await driver.findElement(By.css("#brisiid"));
					await deleteField.click();
					await driver.sleep(2000);
					const delete2Field = await driver.findElement(By.css("#brisiaccid"));
					await delete2Field.click();
					await driver.sleep(2000);
					const loginButtonElement = await driver.findElement(By.css('#login'));
					await loginButtonElement.click();
					const username2Field = await driver.findElement(By.css('#username'));
					await username2Field.sendKeys('da.vinkiiiiiiiii@gmail.com');
					const password2Field = await driver.findElement(By.css('#password'));
					await password2Field.sendKeys('Vinkii.66');
					const submit2Button = await driver.findElement(By.css('#submit'));
					await submit2Button.click();
					await driver.sleep(5000);
				} catch (error) {
					console.error("Test failed:", error);
				} finally {
					// Quit the driver
					console.log("Test je prosao uspjesno.");
					await driver.quit();
				}
				}
				register();
			\end{verbatim}

			\textbf{5. Ispitni slučaj: Prijava štete anonimnog korisnika}
			 \begin{itemize}
				\item \textbf{Ulaz:} Naziv štete, opis štete, adresa štete
				\item \textbf{Očekivani izlaz:} Nakon uspješne prijave izbacit će se prozor da id-jem prijave pomoću kojeg možemo pratiti istu.
				\item \textbf{Koraci:} 
				\\ 1. Klik na gumb Prijavi Štetu!
				\\ 2. Unos podataka prijave
				\\ 3. Klik na gumb Submit
			\end{itemize}

			\begin{verbatim}
				const { By, Key, Builder, until } = require("selenium-webdriver");

				async function prijava() {
				let driver = await new Builder().forBrowser("chrome").build();
				try {
					await driver.get("https://cestafix-fe.onrender.com/");
					const buttonElement = await driver.findElement(By.css('#prijava'));
					await buttonElement.click();
					const nameField = await driver.findElement(By.css('#name'));
					await nameField.sendKeys('Palo stup na kolnik');
					const descriptionField = await driver.findElement(By.css('#description'));
					await descriptionField.sendKeys('Pao STOP znak tamo di hodaju ljudi');
					const addressField = await driver.findElement(By.css('#address'));
					await addressField.sendKeys('Unska 4');
					const submitButton = await driver.findElement(By.css('.confirmButton'));
					await submitButton.click();
					await driver.sleep(40000);
					const confirmButton = await driver.findElement(By.css('.loginbtn'));
					await confirmButton.click();
				} catch (error) {
					console.error("Test failed:", error);
				} finally {
					// Quit the driver
					console.log("Test je prosao uspjesno.");
					await driver.quit();
				}
				}
				prijava();
			\end{verbatim}


			\textbf{6. Ispitni slučaj: Prijava štete prijavljenog korisnika}
			 \begin{itemize}
				\item \textbf{Ulaz:} Email adresa i lozinka korisnika, naziv štete, opis štete, adresa štete
				\item \textbf{Očekivani izlaz:} Nakon uspješne prijave izbacit će se prozor da id-jem prijave pomoću kojeg možemo pratiti istu.
				\item \textbf{Koraci:} 
				\\ 1. Klik na gumb Login/Register
				\\ 2. Unos korisničnih podataka
				\\ 3. Klik na gumb Prijava
				\\ 4. Klik na gumb Prijavi Štetu!
				\\ 5. Unos podataka prijave
				\\ 6. Klik na gumb Submit
			\end{itemize}

			\begin{verbatim}
				const { By, Key, Builder, until } = require("selenium-webdriver");

				async function prijava() {
				let driver = await new Builder().forBrowser("chrome").build();
				try {
					await driver.get("https://cestafix-fe.onrender.com/");
					const buttonElement = await driver.findElement(By.css('#login'));
					await buttonElement.click();
					const usernameField = await driver.findElement(By.css('#username'));
					await usernameField.sendKeys('filip.simunovic@gmail.com');
					const passwordField = await driver.findElement(By.css('#password'));
					await passwordField.sendKeys('Simiklimi.5');
					const submitButton = await driver.findElement(By.css('#submit'));
					await submitButton.click();
					await driver.sleep(5000);
					// prijava
					const button2Element = await driver.findElement(By.css('#prijava'));
					await button2Element.click();
					const nameField = await driver.findElement(By.css('#name'));
					await nameField.sendKeys('Palo stup na kolnik opet');
					const descriptionField = await driver.findElement(By.css('#description'));
					await descriptionField.sendKeys('Pao js jedan 
					STOP znak tamo di hodaju ljudi');
					const addressField = await driver.findElement(By.css('#address'));
					await addressField.sendKeys('Unska 5');
					const submit2Button = await driver.findElement(By.css('.confirmButton'));
					await submit2Button.click();
					await driver.sleep(40000);
				} catch (error) {
					console.error("Test failed:", error);
				} finally {
					// Quit the driver
					console.log("Test je prosao uspjesno.");
					await driver.quit();
				}
				}

				prijava();
			\end{verbatim}


			
			\textbf{7. Ispitni slučaj: Neuspjela registracija korisnika}
			 \begin{itemize}
				\item \textbf{Ulaz:} Ime, prezime, email i \text{neispravna lozinka} (ne zadovoljava uvijet da mora biti dugačka 8 znakova, imati jedno veliko, jedno malo slovo, jedan broj i jedan specijalan znak) korisnika
				\item \textbf{Očekivani izlaz:} Nakon neuspješne registracije korisniku će se ispitati kako lozinka ne zadovoljava sigurnosni uvijet.
				\item \textbf{Koraci:} 
				\\ 1. Klik na gumb Login/Register
				\\ 2. Klik na link "Nemaš račun? Registriraj se!"
				\\ 2. Unos korisničnih podataka
				\\ 3. Klik na gumb Registriraj se
			\end{itemize}

			\begin{verbatim}
					const { By, Key, Builder, until } = require("selenium-webdriver");

				// Test sa neispravnom lozinkom
				async function register() {
					let driver = await new Builder().forBrowser("chrome").build();
				try {
					await driver.get("https://cestafix-fe.onrender.com/");
					const buttonElement = await driver.findElement(By.css('#login'));
					await buttonElement.click();
					const registerElement = await driver.findElement(By.css('#nemaracun'));
					await registerElement.click();
					const imeField = await driver.findElement(By.css('#imeid'));
					await imeField.sendKeys('Da');
					const prezimeField = await driver.findElement(By.css('#prezimeid'));
					await prezimeField.sendKeys('Vinki');
					const emailField = await driver.findElement(By.css('#emailid'));
					await emailField.sendKeys('da.vinki@gmail.com');
					const passwordField = await driver.findElement(By.css('#passwordid'));
					await passwordField.sendKeys('Vinki.5');
					const passwordrepeatField = await driver.findElement
					(By.css('#repatpasswordid'));
					await passwordrepeatField.sendKeys('Vinki.5');
					const submitField = await driver.findElement(By.css('#signupid'));
					await submitField.click();
					await driver.sleep(10000);
				} catch (error) {
					console.error("Test failed:", error);
				} finally {
					// Quit the driver
					console.log("Test je prosao uspjesno.");
					await driver.quit();
				}
				}
				register();
			\end{verbatim}

			\textbf{8. Ispitni slučaj: Neuspjela prijava štete logiranog korisnika}
			 \begin{itemize}
				\item \textbf{Ulaz:} Podatci prijave \textit{bez} podatka o lokaciji
				\item \textbf{Očekivani izlaz:} Nakon što nije unesena nikakva lokacija, korisniku će se u prozoru prikazati ispis "Došlo je do greške, provjerite unos adrese prijave!"
				\item \textbf{Koraci:} 
				\\ 1. Klik na gumb Login/Register
				\\ 2. Unos korisničnih podataka
				\\ 3. Klik na gumb Prijava
				\\ 4. Klik na gumb Prijavi Štetu!
				\\ 5. Unos podataka prijave
				\\ 6. Klik na gumb Submit
			\end{itemize}

			\begin{verbatim}
				const { By, Key, Builder, until } = require("selenium-webdriver");

			// report bez unosa lokacije - error
			async function prijava() {
			let driver = await new Builder().forBrowser("chrome").build();
			try {
				await driver.get("https://cestafix-fe.onrender.com/");
				const buttonElement = await driver.findElement(By.css('#login'));
				await buttonElement.click();
				const usernameField = await driver.findElement(By.css('#username'));
				await usernameField.sendKeys('filip.simunovic@gmail.com');
				const passwordField = await driver.findElement(By.css('#password'));
				await passwordField.sendKeys('Simiklimi.5');
				const submitButton = await driver.findElement(By.css('#submit'));
				await submitButton.click();
				await driver.sleep(5000);
				// prijava
				const button2Element = await driver.findElement(By.css('#prijava'));
				await button2Element.click();
				const nameField = await driver.findElement(By.css('#name'));
				await nameField.sendKeys('Palo stup na kolnik opet');
				const descriptionField = await driver.findElement(By.css('#description'));
				await descriptionField.sendKeys('Pao js jedan STOP znak tamo di hodaju ljudi');
				const submit2Button = await driver.findElement(By.css('.confirmButton'));
				await submit2Button.click();
				await driver.sleep(40000);
			} catch (error) {
				console.error("Test failed:", error);
			} finally {
				// Quit the driver
				console.log("Test je prosao uspjesno.");
				await driver.quit();
			}
			}
			prijava();
			\end{verbatim}

			
			\eject 
		
		\section{Dijagram razmještaja}
			
			\textbf{\textit{dio 2. revizije}}
			
			 \textit{Potrebno je umetnuti \textbf{specifikacijski} dijagram razmještaja i opisati ga. Moguće je umjesto specifikacijskog dijagrama razmještaja umetnuti dijagram razmještaja instanci, pod uvjetom da taj dijagram bolje opisuje neki važniji dio sustava.}
			
			PLACEHOLDER
		
			\begin{figure}[H]
				\includegraphics[scale=0.60]{slike/DR.jpg} %veličina slike u odnosu na originalnu datoteku i pozicija slike
				\centering
				\caption{Dijagram razmještaja}
				\label{fig:DijagramRazmjestaja}
			\end{figure}

			\eject 
		
		\section{Upute za puštanje u pogon}
		
			\textbf{\textit{dio 2. revizije}}\\
		
			 \textit{U ovom poglavlju potrebno je dati upute za puštanje u pogon (engl. deployment) ostvarene aplikacije. Na primjer, za web aplikacije, opisati postupak kojim se od izvornog kôda dolazi do potpuno postavljene baze podataka i poslužitelja koji odgovara na upite korisnika. Za mobilnu aplikaciju, postupak kojim se aplikacija izgradi, te postavi na neku od trgovina. Za stolnu (engl. desktop) aplikaciju, postupak kojim se aplikacija instalira na računalo. Ukoliko mobilne i stolne aplikacije komuniciraju s poslužiteljem i/ili bazom podataka, opisati i postupak njihovog postavljanja. Pri izradi uputa preporučuje se \textbf{naglasiti korake instalacije uporabom natuknica} te koristiti što je više moguće \textbf{slike ekrana} (engl. screenshots) kako bi upute bile jasne i jednostavne za slijediti.}
			
			
			 \textit{Dovršenu aplikaciju potrebno je pokrenuti na javno dostupnom poslužitelju. Studentima se preporuča korištenje neke od sljedećih besplatnih usluga: \href{https://aws.amazon.com/}{Amazon AWS}, \href{https://azure.microsoft.com/en-us/}{Microsoft Azure} ili \href{https://www.heroku.com/}{Heroku}. Mobilne aplikacije trebaju biti objavljene na F-Droid, Google Play ili Amazon App trgovini.}
			
			
			\eject 