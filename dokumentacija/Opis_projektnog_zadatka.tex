\chapter{Opis projektnog zadatka}


\textbf{\textit{dio 1. revizije}}\\

{\color{red}\textit{Na osnovi projektnog zadatka detaljno opisati korisničke zahtjeve. Što jasnije opisati cilj projektnog zadatka, razraditi problematiku zadatka, dodati nove aspekte problema i potencijalnih rješenja. Očekuje se minimalno 3, a poželjno 4-5 stranica opisa.	Teme koje treba dodatno razraditi u ovom poglavlju su:}
\begin{packed_item}
	\item \textit{potencijalna korist ovog projekta}
	\item \textit{postojeća slična rješenja (istražiti i ukratko opisati razlike u odnosu na zadani zadatak). Dodajte slike koja predočavaju slična rješenja.}
	\item \textit{skup korisnika koji bi mogao biti zainteresiran za ostvareno rješenje.}
	\item \textit{mogućnost prilagodbe rješenja }
	\item \textit{opseg projektnog zadatka}
	\item \textit{moguće nadogradnje projektnog zadatka}
\end{packed_item}

\textit{Za pomoć pogledati reference navedene u poglavlju „Popis literature“, a po potrebi konzultirati sadržaj na internetu koji nudi dobre smjernice u tom pogledu.}
\eject
}

Svakodnevno se suočavamo s brojnim problemima na javnim površinama i cestama u našim gradovima. Problemi kao što su vandalizam, oštećenje pločnika, udarne rupe na cestama, smeće i slični problemi predstavljaju potencijalnu opasnost za građane te poprilično narušavaju kvalitetu njihovog svakodnevnog života. Neadekvatna briga o oštećenjima i njihovoj sanaciji često ostavlja građane u vrlo nepovoljnoj situaciji kada iste žele prijaviti vlastima.

Kako bi suzbili osjećaj nemoći građana i poboljšali kvalitetu života cjelokupne zajednice, potrebno je omogućiti adekvatnu prijavu oštećenja na javnim površinama.

Glavni cilj ovog projekta je razviti programsku podršku za stvaranje web aplikacije “Unesi ime aplikacije koje moramo smisliti” za dojavu oštećenja i drugih problema na cestama, parkovima, javnim ustanovama i ostalim javnim mjestima u svrhu olakšavanja dojave, kategorizacije te u konačnici rješenja prijavljenih problema.

Ideja je omogućiti građanima da na što jednostavniji način prijavljuju oštećenja i probleme na javnim površinama i cestama svojih gradova te tako pomognu lokalnim vlastima da pravodobno reagiraju na nastale probleme. S obzirom na velik broj javih ustanova i njihovih odjeljaka koji se bave različitim problemima, teško je pratiti tko je nadležan za koju vrstu problema i nad kojim područjem. Sustav bi automatski odredio nadležno tijelo prema kategoriji prijave i drugim parametrima te proslijedio prijavu na obradu. Ključno je da proces prijave bude jednostavan i brz.

Prilikom pokretanja sustava prikazuje se početna stranica na kojoj se nalazi karta sa označenim lokacijama već podnesenih prijava raznovrsnih oštećenja javnih površina, nekoliko informacija o samoj aplikaciji te navigacijski izbornik putem kojeg se pristupa registraciji/prijavi u sustav te podnošenju prijava i provjeri statusa već podnesenih prijava.

Svakom neregistriranom korisniku omogućeno je prijavljivanje u sustav s postojećim računom, prilikom čega je potrebno upisati korisničko ime i lozinku, ili kreiranje novog računa prilikom čega je potrebno odabrati korisničko ime i lozinku te upisati valjanu email adresu.

Također, postoji opcija anonimnog podnošenja prijave na temelju koje se status podnesene prijave prati putem jedinstvenog broja prijave bez iznošenja osobnih podataka.

Registrirani korisnik može pregledavati i mijenjati svoje osobne podatke i izbrisati svoj korisnički račun te postoji mogućnost naknadne dodjele prava službenika ili administratora stupanjem u kontakt naveden pri dnu stranice.

Kako bi olakšala proces podnošenja prijave, aplikacija omogućuje građanima da brzo i precizno dokumentiraju probleme. Svaka prijava uključuje naziv problema, kratki opis, geografske koordinate problema, jedinstveni broj prijave te opcionalno fotografije kako bi se problem što bolje opisao. Građani mogu odabirom lokacije na karti, unosom najbliže adrese ili iz meta podataka slika priloženih uz prijavu precizno definirati lokaciju problema na koji su naišli.

Sustav također olakšava identifikaciju vremenski bliskih prijava na istoj lokaciji kako bi se građani mogli povezati na već postojeće prijave sličnih problema, čime se smanjuje dupliciranje prijava i ubrzava proces njihovog rješavanja.

Podnesene prijave obrađuju određeni gradski uredi, a svaka promjena u statusu prijave vidljiva je prijavitelju i svim ostalim korisnicima koji su povezani s tom lokacijom. Gradski uredi također imaju mogućnost objediniti nepovezane prijave na istoj lokaciji, što je vidljivo prijaviteljima.

Kako bi gradskim uredima olakšali razvrstavanje i obrađivanje prijava, sustav dopušta prijave za različite kategorije problema.

Sve podnesene prijave su javno dostupne i mogu se grupirati prema temi i lokaciji, što omogućuje građanima i vlastima da prate i analiziraju probleme u stvarnom vremenu.

Korisnik s pravom službenika ima mogućnost ažuriranja statusa prijava i pregled profila prijavitelja (ili jedinstvenog broja prijave u slučaju anonimnog prijavitelja) među kojima odabire relevantne prijave. Nakon što sustav proslijedi prijavu u nadležni gradski ured, službenik tog ureda može izabrati problem i krenuti raditi na njegovu rješenju.

Administrator sustava ima najveće ovlasti među koje pripada i pristup bazi s popisom registriranih korisnika i njihovim podacima te mogućnost brisanja korisnika kao i dodjeljivanje administratorskih prava i prava službenika drugim korisnicima.

Statistika o statusima prijava, poput vremena potrebnog za rješavanje problema, također se obrađuje u stvarnom vremenu kako bi poboljšala učinkovitost sustava.

Posebna je pozornost posvećena zaštiti svih osobnih podataka te osiguranju korisničkog iskustva koje će biti jednostavno i intuitivno, kako bi se potaknulo građane na aktivno sudjelovanje u poboljšanju svojih zajednica.

Jedan od primjera sličnog sustava koji se aktivno upotrebljava u Hrvatskoj je "Gradsko Oko". Projekt je pokrenut u kolovozu 2017. godine u svrhu prijave komunalnih problema na području grada Bjelovara, a od tad se proširio na još 11 gradova i općina te na prijavu problema na moru.

Iako je projekt sličan našem razlikuje se u nekoliko točaka: ne dozvoljava anonimne prijave, ne sadrži mogućnost pregleda statistike i ne provjerava sličnost vremenski bliskih prijava u svrhu grupiranja.



